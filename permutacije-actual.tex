\documentclass[12pt,a4paper,reqno]{amsart}
% ukazi za delo s slovenscino -- izberi kodiranje, ki ti ustreza
\usepackage[slovene]{babel}
%\usepackage[cp1250]{inputenc}
\usepackage[T1]{fontenc}
\usepackage[utf8]{inputenc}
\usepackage{amsmath,amssymb,amsfonts}
\usepackage{url}
%\usepackage[normalem]{ulem}
\usepackage[dvipsnames,usenames]{color}

% ne spreminjaj podatkov, ki vplivajo na obliko strani
\textwidth 15cm
\textheight 24cm
\oddsidemargin.5cm
\evensidemargin.5cm
\topmargin-5mm
\addtolength{\footskip}{10pt}
\pagestyle{plain}
\overfullrule=15pt % oznaci predlogo vrstico


% ukazi za matematicna okolja
\theoremstyle{definition} % tekst napisan pokoncno
\newtheorem{definicija}{Definicija}[section]
\newtheorem{primer}[definicija]{Primer}
\newtheorem{opomba}[definicija]{Opomba}

\theoremstyle{plain} % tekst napisan posevno
\newtheorem{lema}[definicija]{Lema}
\newtheorem{izrek}[definicija]{Izrek}
\newtheorem{trditev}[definicija]{Trditev}
\newtheorem{posledica}[definicija]{Posledica}


% za stevilske mnozice uporabi naslednje simbole
\newcommand{\R}{\mathbb R}
\newcommand{\N}{\mathbb N}
\newcommand{\Z}{\mathbb Z}
\newcommand{\C}{\mathbb C}
\newcommand{\Q}{\mathbb Q}


% naslednje ukaze ustrezno popravi
\newcommand{\program}{Matematika} % ime studijskega programa: Matematika/Finan"cna matematika
\newcommand{\imeavtorja}{Katarina Černe} % ime avtorja
\newcommand{\imementorja}{izred.~prof.~dr.~Barbara Drinovec Drnovšek} % akademski naziv in ime mentorja
\newcommand{\naslovdela}{Permutacije in konvergenca številskih vrst}
\newcommand{\letnica}{2016} %letnica diplome

\begin{document}
% od tod do povzetka ne spreminjaj nicesar
\thispagestyle{empty}
\noindent{\large
UNIVERZA V LJUBLJANI\\[1mm]
FAKULTETA ZA MATEMATIKO IN FIZIKO\\[5mm]
\program\ -- 1.~stopnja}
\vfill

\begin{center}{\large
\imeavtorja\\[2mm]
{\bf \naslovdela}\\[10mm]
Delo diplomskega seminarja\\[1cm]
Mentor: \imementorja}
\end{center}
\vfill

\noindent{\large
Ljubljana, \letnica}
\pagebreak

\thispagestyle{empty}
\tableofcontents
\pagebreak

\section{Uvod}

\emph{Številska vrsta} $\sum^{\infty}_{n=1}a_n$ je neskončna vsota členov zaporedja realnih števil\\ $\{a_{n}\}_{n\in \N}$. \emph{Zaporedje delnih vsot} številske vrste je definirano kot zaporedje $\{S_{n}\}_{n\in \N}$ s členi $S_i = a_1 + a_2 + \ldots + a_i.$ Pravimo, da številska vrsta $\sum^{\infty}_{n=1}a_n$ \emph{konvergira}, če konvergira zaporedje njenih delnih vsot $\{S_n\}^{\infty}_{n=1}$. Vsota te vrste je enaka limiti zaporedja delnih vsot. Konvergentne vrste ločimo na \emph{absolutno konvergentne} in \emph{pogojno konvergentne}. Vrsta $\sum^{\infty}_{n=1}a_n$ \emph{konvergira absolutno}, če konvergira vrsta $\sum^{\infty}_{n=1}|a_n|$, pogojno pa konvergira, če vrsta $\sum^{\infty}_{n=1}a_n$ konvergira, vrsta $\sum^{\infty}_{n=1}|a_n|$ pa divergira, pa rečemo, da je vrsta $\sum^{\infty}_{n=1}a_n$ \emph{pogojno konvergentna}. V zvezi s konvergenco vrst je potrebno omeniti tudi dva pomembna izreka.

\begin{izrek}\label{izr:absolutno}
Naj bo $\sum^{\infty}_{n=1}a_n$ absolutno konvergenta. Potem za vsako permutacijo $\pi$ konvergira tudi $\sum^{\infty}_{n=1}a_{\pi (n)}$ in velja $$\sum^{\infty}_{n=1}a_n = \sum^{\infty}_{n=1}a_{\pi (n)}.$$
\end{izrek}

\begin{izrek}\label{izr:riemann}
Naj bo $\sum^{\infty}_{n=1}a_n$ pogojno konvergentna vrsta. Potem za vsako število $A\in \R \cup \{ \pm \infty \}$ obstaja taka permutacija $\pi$, da je $$\sum^{\infty}_{n=1}a_{\pi (n)} = A.$$
\end{izrek}

Vidimo torej, da lahko v absolutno konvergentnih vrstah poljubno premešamo člene, ne da bi kakorkoli vplivali na vsoto ali konvergenco vrste, medtem ko za pogojno konvergentne vrste obstajajo permutacije, ki spremenijo vrstni red členov tako, da lahko dobljena vrsta konvergira h kateremukoli realnemu številu, ali pa celo divergira. Pri tem se pojavi vprašanje, kakšne so te permutacije. Se jih da kako klasificirati? Imajo kakšne skupne lastnosti? V nadaljevanju se bomo ukvarjali predvsem s permutacijami na členih pogojno konvergentnih vrst, ki ohranjajo konvergenco. Primer takih permutacij so t.~i.\ $\lambda$-permutacije.

\subsection{Permutacije z lastnostjo $\lambda$}

\begin{definicija}
Permutacijo naravnih števil $\sigma$ imenujemo \emph{$\lambda$-permutacija} oz. \emph{permutacija z lastnostjo $\lambda$}, če velja:
\begin{itemize}
\item[(1)] Če vrsta $\displaystyle\sum^{\infty}_{n=1}a_n$ konvergira, potem konvergira tudi $\displaystyle\sum^{\infty}_{n=1}a_{\sigma (n)}.$
\item[(2)] Obstaja vsaj ena divergentna vrsta $\displaystyle\sum^{\infty}_{n=1}b_n$, da vrsta $\displaystyle\sum^{\infty}_{n=1}b_{\sigma (n)}$ konvergira.
\end{itemize}
\end{definicija}

\begin{opomba}
V nadaljevanju bomo divergentne vrste, za katere obstaja kakršnakoli permutacija, ki vrstni red elementov spremeni tako, da novonastala vrsta konvergira, imenovali kar \emph{pogojno divergentne vrste}.
\end{opomba}

Naravno se ob definiciji $\lambda$-permutacij pojavi kar nekaj vprašanj. Najprej se lahko vprašamo, kako težko je skonstruirati tako permutacijo, oziroma ali kakšna taka permutacija sploh obstaja. Nadalje, koliko takih permutacij obstaja. Kaj še lahko povemo o permutacijah z lastnostjo $\lambda$ in množici vseh $\lambda$-permutacij? Eno izmed pomembnejših vprašanj, ki se porajajo ob raziskovanju lastnosti množice $\lambda$-permutacij, je vprašanje njene omejenosti. Če je ta množica navzgor omejena, namreč obstaja optimalna $\lambda$-permutacija, ki ustvari največ konvergentnih vrst. Če je to res in če je ta optimalna perutacija različna od identitete, bi bilo bolje, da bi, kar se tiče seštevanja vrst, namesto običajnega uporabljali drugačno zaporedje naravnih števil, torej zaporedje, ki ga narekuje dobljena optimalna permutacija? Nazadnje pa se lahko vprašamo še, ali nam $\lambda$ permutacije lahko pomagajo pri vsaki pogojno divergentni vrsti oziroma ali za vsako pogojno divergentno vrsto $\sum^{\infty}_{n=1}a_n$ obstaja $\lambda$-permutacija $\sigma$, da $\sum^{\infty}_{n=1}a_{\sigma (n)}$ konvergira. To so vprašanja, na katera bo odgovorjeno v tej diplomski nalogi.

\section{Primer konstrukcije $\lambda$-permutacije} \label{sec:konstrukcija}

Oglejmo si primer konstrukcije permutacije z lastnostjo $\lambda$. S tem bomo tudi pokazali, da množica $\lambda$-permutacij ni prazna.
Naravna števila najprej razvrstimo v bloke naraščajoče velikosti na naslednji način:

\begin{equation}
1 \quad 2\: 3\: 4 \quad 5\: 6\: 7\: 8\: 9 \quad 10\: 11\: 12\: 13\: 14\: 15\: 16 \quad 17\: 18\: 19\: 20\: 21\: 22\: 23\: 24\: 25\: \ldots 
\label{eq:1}
\end{equation}

Bloki imajo liho število elementov. Moč vsakega bloka je za $2$ večja od moči predhodnega bloka. Sedaj pa vsak blok preoblikujemo tako, da na začetek postavimo število na sredini bloka. Na drugem mestu bo število, ki je za $1$ večje od srednjega števila, sledi število, ki je od srednjega za $1$ manjše, potlej število za $2$ večje od srednjega, in tako dalje dokler ne preuredimo vseh števil v bloku. Dobimo naslednje zaporedje:

\begin{equation}  
1 \quad 3\: 4\: 2 \quad 7\: 8\: 6\: 9\: 5 \quad 13\: 14\: 12\: 15\: 11\: 16\: 10 \quad 21\: 22\: 20\: 23\: 19\: 24\: 18\: 25\: 17 \ldots 
\label{eq:2}
\end{equation}

Permutacijo, ki na zaporedje naravnih števil deluje na zgoraj opisani način, označimo s $\sigma$. 

Permutacijo $\sigma$ uporabimo na členih številske vrste $\sum^{\infty}_{n=1}a_n$. Njene delne vsote bomo označili s $S_n$. Opazimo lahko, da je vsaka delna vsota $\tilde{S_n}$ permutirane vrste $\sum^{\infty}_{n=1}a_{\sigma(n)}$ linearna kombinacija največ treh delnih vsot prvotne vrste. Natančneje, velja, da je 
\begin{equation}
\tilde{S_N}=S_{q^2}+(S_B-S_A).\label{eq:3} 
\end{equation}
Tu je $q^2$ največji popolni kvadrat, manjši od $N$, in $A \geq q^2$, $B \geq q^2$ ter $A \leq B$. Vsak blok v zaporedju (\ref{eq:1}) se namreč konča s popolnim kvadratom, tako da so indeksi členov v delni vsoti $S_{q^2}=a_1+a_2+\ldots+a_{q^2}$ vsa števila iz blokov od prvega do vključno tistega, ki se konča s $q^2$. Ker se permutiranje členov izvaja samo znotraj blokov, bo imela delna vsota $\tilde{S_{q^2}}$ enake člene kot $S_{q^2}$, le v drugačnem vrstnem redu. 
Delna vsota $S_4$ na primer izgleda tako: $$S_4 = a_1+a_2+a_3+a_4,$$ medtem ko je $\tilde{S}_4$ taka: $$\tilde{S}_4=a_{\sigma(1)}+a_{\sigma(2)}+a_{\sigma(3)}+a_{\sigma(4)}=a_1+a_3+a_4+a_2.$$ Vidimo, da sta obe delni vsoti enaki, saj vsebujeta vse člene z indeksi iz prvih dveh blokov. Za delne vsote  permutirane vrste torej velja: $$\tilde{S_n}=\tilde{S_{q^2}}+Q=S_{q^2}+Q,$$ kjer je $q^2$ največji popolni kvadrat, manjši od $n$, $Q$ pa vsebuje člene z indeksi, večjimi od $q^2$. Po kontrukciji permutacije $\sigma$ velja, da če $Q$ vsebuje člena $a_C$ in $a_B$, kjer je $C < B$, potem bo vseboval tudi vse člene $a_i$, kjer je $C<i<B$. Torej lahko zapišemo $Q=S_B-S_{C-1}$, od koder sledi enačba (\ref{eq:3}).

Po Cauchyjevem kriteriju za konvergenco zaporedij vrsta $\sum^{\infty}_{n=1}a_n$ konvergira natanko tedaj, ko za vsak $\epsilon >0$ obstaja $n_0 \in \N$, da za vsak par indeksov $m,n\in \N \textrm{, }m>n\geq n_0$ velja $$|S_m-S_n|<\epsilon$$. Ker je vsaka delna vsota vrste $\sum^{\infty}_{n=1}a_{\sigma(n)}$ končna linearna kombinacija delnih vsot vrste $\sum^{\infty}_{n=1}a_{n}$, bo Cauchyjev kriterij, če bo veljal za vrsto $\sum^{\infty}_{n=1}a_{n}$, veljal tudi za $\sum^{\infty}_{n=1}a_{\sigma(n)}$. S tem je zadoščeno prvi točki iz definicije $\lambda$-permutacije. %natančnejša razlaga?

Da pokažemo, da je $\sigma$ res $\lambda$-permutacija, moramo poiskati še pogojno divergentno vrsto, iz katere bo permutacija $\sigma$ ustvarila pogojno konvergentno vrsto. Primer take divergente vrste je:
$$1+\left (-\frac{1}{2}-\frac{1}{2}+\frac{1}{2}\right )+\left (\frac{1}{3}+\frac{1}{3}+\frac{1}{3}-\frac{1}{3}-\frac{1}{3} \right )+\left (-\frac{1}{4}-\frac{1}{4}-\frac{1}{4}-\frac{1}{4}+\frac{1}{4}+\frac{1}{4}+\frac{1}{4}\right )+$$ $$+\left (\frac{1}{5}+\frac{1}{5}+\frac{1}{5}+\frac{1}{5}+\frac{1}{5}-\frac{1}{5}-\frac{1}{5}-\frac{1}{5}-\frac{1}{5} \right )+\dots $$
%kako naj razložim, da divergira?

Ko na zgornji vrsti uporabimo permutacijo $\sigma$, dobimo:
$$1-\frac{1}{2}+\frac{1}{2}-\frac{1}{2}+\frac{1}{3}-\frac{1}{3}+\frac{1}{3}-\frac{1}{3}+\frac{1}{3}-\frac{1}{4}+\frac{1}{4}-\frac{1}{4}+\frac{1}{4}-\frac{1}{4}+\frac{1}{4}-\frac{1}{4}+\frac{1}{5}-\frac{1}{5} \pm \dots$$
Dobljena vrsta konvergira po Leibnizevem testu za alternirajoče vrste, ki pravi, da alternirajoča vrsta konvergira, kadar ima zaporedje absolutnih vrednosti njenih členov limito 0. %bolj natančna razlaga?

Našli smo pogojno divergentno vrsto, ki jo $\sigma$ spremeni v konvergentno. Permutacija $\sigma$ torej zadošča obema točkama v definiciji $\lambda$-permutacije. S tem smo dokazali, da obstaja vsaj ena $\lambda$-permutacija. Je to edina $\lambda$-permutacija ali obstaja še kakšna? Da odgovorimo na to vprašanje, si moramo najprej ogledati nekaj lastnosti $\lambda$-permutacij.

\section{Lastnosti $\lambda$-permutacij}
%tu je treba najbrž povedati, kaj se bo opazovalo v tem poglavju in zakaj je dobro gledati te lastnosti
\subsection{Ohranjanje konvergence in vsote}

\begin{definicija}
Pravimo, da permutacija \emph{ohranja konvergenco}, kadar iz konvergence vrste $\sum^{\infty}_{n=1}a_n$ sledi konvergenca vrste $\sum^{\infty}_{n=1}a_{\sigma (n)}$, in \emph{ohranja vsoto}, kadar velja, da je $\sum^{\infty}_{n=1}a_n=\sum^{\infty}_{n=1}a_{\sigma (n)}$.
\end{definicija}

Za lažje obravnavanje lastnosti $\lambda$-permutacij vpeljimo notacijo %ali mora biti to definicija??

$$[c,d]_{\Z}=\{x \in \Z^+ ; c \leq x \leq d \},$$

kjer sta $c$ in $d$ naravni števili, za kateri velja $c\leq d$. 

Z notacijo $[2,5]$ na primer označimo množico $\{2,3,4,5\}$.

Naj bo sedaj $\sigma$ neka permutacija naravnih števil in $n$ naravno število. Zapišemo lahko:

\begin{equation} \label{eq:4}
\{ \sigma (1), \sigma (2), \ldots \sigma (n) \} = [c^n_1, d^n_1]_{\Z} \cup [c^n_2, d^n_2]_{\Z} \cup \cdots \cup [c^n_{b_n}, d^n_{b_n}]_{\Z},
\end{equation} 

pri čemer je $c^n_i \leq d^n_i$ in $c^n_{i+1} \geq d^n_i +2$. Z $b_n$ smo označili število blokov oblike $[c^n_i, d^n_i]_{\Z}$, ki sestavljajo zgornjo unijo. Za permutacijo $\sigma$ lahko definiramo zaporedje $\{b_n\}^{\infty}_{n=1}$, ki ga imenujemo \emph{zaporedje števil blokov}. %posebej definicija zaporedja števil blokov?
V nadaljevanju bomo rabili še notacijo $$M_n=d^n_{b_n}+1=max(\{ \sigma (1), \sigma (2), \ldots \sigma (n) \})+1,$$ kjer je $d^n_{b_n}$ število, s katerim se konča zadnji blok v notaciji \ref{eq:4}.

\begin{primer}
Vzemimo permutacijo $\sigma$, ki smo jo skonstruirali v poglavju \ref{sec:konstrukcija}, torej 
$$\sigma = \bigg(\begin{matrix}
    1 & 2 & 3 & 4 & 5 & 6 & \cdots   \\   1 & 3 & 4 & 2 & 7 & 8 & \cdots
  \end{matrix}\bigg).$$
Potem lahko zapišemo $$\{\sigma(1), \sigma(2), \ldots, \sigma(8) \} = \{ 1,3,4,2,7,8\} = [1,4]_{\Z} \cup [7,8]_{\Z}.$$ V tem primeru je število blokov $b_8=2$, saj lahko $\{\sigma(1), \sigma(2), \ldots, \sigma(8) \}$ zapišemo kot unijo najmanj dveh blokov oblike  $[c^n_i, d^n_i]_{\Z}$.
Oglejmo si še zaporedje števil blokov za permutacijo $\sigma$. Očitno je $b_1=1$. Ker je $$\{\sigma(1), \sigma(2) \} = \{ 1,3\} = [1,1]_{\Z} \cup [3,3]_{\Z},$$ je $b_2=2$. Za $n=3$ velja $$\{\sigma(1), \sigma(2), \sigma(3) \} = \{ 1,3,4\} = [1,1]_{\Z} \cup [3,4]_{\Z},$$ torej je $b_3$=2, in tako dalje. Zaporedje števil blokov za permutacijo $\sigma$ je torej videti tako: $$1\:2\:2\:1\:2\:2\:\ldots.$$
\end{primer}

Zgoraj vpeljane notacije bodo zelo uporabne pri dokazu sledeče trditve, ki govori o povezavi med ohranjanjem konvergence in vsote.

\begin{trditev} \label{trd:ohranjanje}
Naj bo $\sigma$ permutacija naravnih števil. Naslednje trditve so ekvivalentne:
\begin{enumerate}
\item permutacija $\sigma$ ohranja konvergenco \label{itm:1}
\item zaporedje števil blokov $\{b_n\}^{\infty}_{n=1}$ je omejeno \label{itm:2}
\item permutacija $\sigma$ ohranja vsoto \label{itm:3}
\end{enumerate}
\end{trditev}

\begin{proof}
Da iz (\ref{itm:3}) sledi (\ref{itm:1}) je očitno.

Dokažimo sedaj, da iz (\ref{itm:1}) sledi (\ref{itm:2}).
Naj bo $\sigma$ neka permutacija naravnih števil, ki ohranja konvergenco. Dokaza se lotimo s protislovjem. Predpostavimo, da je zaporedje števil blokov $\{b_n\}^{\infty}_{n=1}$ permutacije $\sigma$ neomejeno. Poiskali bomo tako konvergentno vrsto $\sum^{\infty}_{n=1}a_n$, da bo $\sum^{\infty}_{n=1}a_{\sigma (n)}$ divergentna, kar je v protislovju s predpostavko, da $\sigma$ ohranja konvergenco. 

Izberimo tak $n_1\in \N$, da je $c_1^{n_1}=1$, kjer je  $c_1^{n_1}$ število,s katerim se začne prvi blok v izrazu $\{ \sigma (1), \sigma (2), \ldots \sigma (n_1) \} = [c^{n_1}_1, d^{n_1}_1]_{\Z} \cup [c^{n_1}_2, d^{n_1}_2]_{\Z} \cup \cdots \cup [c^{n_1}_{b_n}, d^{n_1}_{b_n}]_{\Z}$. Drugače rečeno, naj bo $1\in \{ \sigma (1), \sigma (2), \ldots \sigma (n_1) \}$. Za tak $n_1$ za vsak $k\in \N$, za katerega velja $1 \leq k \leq M_{n_1}$ definiramo člen $a_k$ zaporedja $\{a_n\}^{M_{n_1}}_{n=1}$ na naslednji način:

$$a_k = 
\left\{ 
\begin{array}{ccc}
1&;&k=d_i^{n_1}\textrm{ za nek i, }1\leq i \leq b_{n_1}\\
-1&;&k=d_i^{n_1}+1\textrm{ za nek i, }1\leq i \leq b_{n_1}\\
0&;&\textrm{sicer}\\
\end{array}
\right. 
$$

Če imamo na primer permutacijo $\sigma$ kot v poglavju \ref{sec:konstrukcija} in $n_1=6$, potem je $$\{ \sigma (1), \sigma (2), \ldots \sigma (6) \} = \{1, 3, 4, 2, 7, 8 \}= [1,4]_{\Z} \cup [7,8]_{\Z},$$ za zaporedje $\{a_n\}^{9}_{n=1}$ pa velja, da je $$a_4=1,\ a_5=-1,\ a_8 =1\textrm{ in }a_9=-1,$$ vsi ostali členi pa so enaki nič.

Oglejmo si nekaj značilnosti tega zaporedja.%katerega 
Najprej opazimo, da velja enakost $$\sum_{k=1}^{n_1}a_{\sigma(k)}=b_{n_1}\geq 1$$. Členi zaporedja $\{a_n\}^{M_{n_1}}_{n=1}$, ki so enaki $-1$ namreč zagotovo niso vključeni v to vsoto. Členi, enaki $a_k=-1$, so namreč tisti, za katere je $k=d_i^{n_1}+1$, $d_i^{n_1}+1$ pa za noben $i$ ni element množice $\{ \sigma (1), \sigma (2), \ldots \sigma (n_1) \}$, saj mora biti med $d_i^{n_1}$ in $c_{i+1}^{n_1}$ vsaj eno število, ki ni v množici $\{ \sigma (1), \sigma (2), \ldots \sigma (n_1) \}$. V nasprotnem primeru bi namesto $[c_i^{n_1},d_i^{n_1}] \cup [c_{i+1}^{n_1},d_{i+1}^{n_1}]$ v izrazu \ref{eq:4} pisali kar $[c_i^{n_1},d_{i+1}^{n_1}]$. 

Členi zaporedja $\{a_n\}^{M_{n_1}}_{n=1}$, ki so enaki $a_k=1$, so zagotovo vsi vključeni v zgornjo vsoto, saj zanje velja, da je $k=d_i^{n_1}$ za $1\leq i\leq b_{n_1}, d_i^{n_1}$ pa je za vsak tak $i$ element množice $\{ \sigma (1), \sigma (2), \ldots \sigma (n_1) \}$. Ta množica torej vsebuje ravno $b_{n_1}$ elementov oblike $d_i^{n_1}$, torej bo v vsoti $\sum_{k=1}^{n_1}a_{\sigma(k)}$ ravno $b_{n_1}$ členov, ki bodo enaki $1$, vsi ostali pa bodo ničelni. Od tod torej sledi $\sum_{k=1}^{n_1}a_{\sigma(k)}=b_{n_1}$, da je $b_{n_1}\geq 1$, pa je očitno.

Poleg tega vidimo še, da je $$\sum_{k=1}^{M_{n_1}}a_k=0.$$ Členi zaporedja $\{a_n\}^{M_{n_1}}_{n=1}$, ki so enaki $1$ in $-1$ namreč vselej nastopajo v parih. Če je $a_i=1$, potem je $a_{i+1}=-1$. Zadnji člen vsote je $a_{M_{n_1}}=-1$, saj je $M_{n_1}=d_{b_{n_1}}^{n_1}+1$. V vsoti $\sum_{k=1}^{M_{n_1}}a_k=0$ je tako enako število členov, enakih $1$ in členov, enakih $-1$, ki se med sabo odštejejo, vsi ostali členi v vsoti pa so ničelni.

Na podoben način pokažemo tudi, da je $$\sum_{k=1}^{n}a_k=0\textrm{ ali }1,$$ kjer za $n$ velja $1\geq n\geq M_{n_1}$. Za tako vsoto namreč obstajajo tri možnosti: ali je zadnji člen enak $0$ ali $1$ ali pa $-1$. Če je enak $0$ ali $-1$, bo število členov, enakih $1$, enako številu členov, enakih $-1$ in bo vsota enaka nič. Če pa je zadnji člen enak $1$, bo število členov, enakih $1$, za eno večje od števila členov, enakih $-1$, ($-1$ namreč vedno sledi takoj za $1$), torej bo vsota enaka $1$.

Do sedaj smo skonstruirali zaporedje $\{a_n\}^{M_{n_1}}_{n=1}$, radi pa bi imeli zaporedje $\{a_n\}^{\infty}_{n=1}$. Preostanek zaporedja konstruiramo rekurzivno: za vsak $j\in \N,\ j>1$ moramo izbrati $n_j$, da je $M_{n_j}>M_{n_{j-1}}$ in za vsak $k$, ki ustreza $M_{n_{j-1}}<k\leq M_{n_j}$ moramo definirati $a_k$ tako, da bo veljalo $\sum_{k=1}^{M_{n_j}}a_k=0$.

Recimo, da je $j>1$ in da smo v prejšnjem koraku že izbrali tak $n_{j-1}$ in take $a_k$ za $1\leq k\leq M_{n_j}$, da je veljalo $\sum_{k=1}^{M_{n_{j-1}}}a_k=0$. Ker smo predpisali, da je zaporedje števil blokov $\{b_n\}^{\infty}_{n=1}$ permutacije $\sigma$ neomejeno, lahko izberemo tak $n_j$, da je $d_1^{n_j}>M_{n_{j-1}}$ in $b_{n_j}\geq j^2$. %zakaj to naredimo
Definirajmo sedaj $a_k$ za $M_{n_{j-1}}<k\leq M_{n_j}$:

$$a_k = 
\left\{ 
\begin{array}{ccc}
\frac{1}{j}&;&k=d_i^{n_j}\textrm{ za nek i, }1\leq i \leq b_{n_1}\\
-\frac{1}{j}&;&k=d_i^{n_j}+1\textrm{ za nek i, }1\leq i \leq b_{n_1}\\
0&;&\textrm{sicer}\\
\end{array}
\right. 
$$

%Ker smo izbrali $n_j$ tako, da je $d_1^{n_j}>M_{n_{j-1}}$, velja $\{ \sigma (1), \sigma (2), \ldots \sigma (n_j) \} = [1, d_1^{n_j}]_{\Z} \cup [c_2^{n_j}, d_2^{n_j}]_{\Z} \cup \cdots \cup [c^{n_j}_{b_n}, d^{n_j}_{b_n}]_{\Z}$, kjer $ [1, d_1^{n_j}]_{\Z}$ vsebuje vsa števila med $1$ in $M_{n_j}$.
%ta del dokaza mi ni čisto jasen
Podobno kot pri $j=1$ tudi tu vidimo, da je $$\sum_{k=1}^{n_j}a_{\sigma(k)}=\frac{b_{n_j}}{j}\geq j\textrm{, }\sum_{k=1}^{M_{n_1}}a_k=0$$ in $$\sum_{k=1}^{n}a_k=0\textrm{ ali }\frac{1}{j}\textrm{ za }M_{n_{j-1}}<n<M_{n_j}.$$%zakaj rabim ta zadnji del

Če sedaj pošljemo $j$ proti neskončno, dobimo, $\sum_{k=1}^{\infty}a_k=0$. Ta vrsta torej konvergira. Po drugi strani pa $\sum_{k=1}^{M_{n_1}}a_{\sigma(k)}$ divergira, saj je $\sum_{k=1}^{n_j}a_{\sigma(k)}\geq j$ za vsak $j$. Torej $\sigma$ ne ohranja konvergence, kar je v nasprotju z začetno predpostavko. Sledi, da je zaporedje števil blokov $\{b_n\}^{\infty}_{n=1}$ omejeno.

Zadnja implikacija, ki jo moramo dokazati, je, da iz (\ref{itm:2}) sledi (\ref{itm:3}).
Imejmo permutacijo naravnih števil $\sigma$ in naj bo $\{b_n\}_{n=1}^{\infty}$ njeno zaporedje števil blokov, ki je omejeno. Naj bo $\{a_k\}_{k=1}^{\infty}$ tako zaporedje, da velja $$\sum_{k=1}^{\infty}a_k=L\in \R.$$ Pokazati hočemo, da pri takih pogojih $\sigma$ ohranja vsoto, torej, da je $$\sum_{k=1}^{\infty}a_k=\sum_{k=1}^{\infty}a_{\sigma(k)}=L.$$ 
Označimo s $$S(n)=\sum_{k=1}^{n}a_k$$ delne vsote vrste $\sum_{k=1}^{\infty}a_k$. Iz definicije vsote vrste sledi, da je $$\lim_{n\to \infty}S(n)=L.$$ Oglejmo si sedaj delovanje permutacije $\sigma$. Naj bo $n$ tako velik, da bo v zapisu $\{ \sigma (1), \sigma (2), \ldots \sigma (n) \} = [c^n_1, d^n_1]_{\Z} \cup [c^n_2, d^n_2]_{\Z} \cup \cdots \cup [c^n_{b_n}, d^n_{b_n}]_{\Z}$ veljalo, da je $c^n_1$ enak $1$. Potem je 
\begin{equation} \label{eq:5}
\sum_{k=1}^{n}a_{\sigma(k)}=\sum_{k=1}^{d_1^n}a_k+\sum_{i=2}^{b_n}\sum_{k=c_i^n}^{d_i^n}a_k = S(d_1^n)+\sum_{i=2}^{b_n}(S(d_i^n)-S(c_i^n-1)).
\end{equation}
Ko pošljemo $n$ proti neskončno, gre $d_1^n$ proti neskončno. Torej gre $S(d_1^n)$ proti $L$. Oglejmo si še limito izraza $(S(d_i^n)-S(c_i^n-1))$. Ko gre $d_1^n$ proti neskončno, se blok  $[c^n_1, d^n_1]_{\Z}$ veča in zavzema vedno več števil, posledično pa se ostali bloki manjšajo. Torej sta števili $d_i^n$ in $c_i^n-1$, ko gre $n$ proti neskončno, vedno bližje eno drugemu. Zato gre $(S(d_i^n)-S(c_i^n-1))$ proti $0$. Dobimo torej $$\sum_{k=1}^{\infty}a_{\sigma(k)}=\lim_{n \to \infty}\sum_{k=1}^{\infty}a_{\sigma(k)}=L.$$
%natančnejša razlaga zadnjega dela
\end{proof}

Enakost \ref{eq:5}, ki smo jo uporabili v zgornjem dokazu, ima naslednjo posledico:
\begin{posledica}
Naj bo $\sigma$ permutacija, ki ohranja konvergenco in $\sum^{\infty}_{n=1}a_n$ konvergentna vrsta. Potem obstaja tako število $B\in \mathbb{N}$, da je vsaka delna vsota vrste $\sum^{\infty}_{n=1}a_{\sigma(n)}$ linearna kombinacija največ $B$ delnih vsot prvotne vrste.
\end{posledica}

\begin{proof}
Naj bo $\sigma$ permutacija, ki ohranja konvergenco. Potem je njeno zaporedje števil blokov $\{b_n\}_{n=1}^{\infty}$ omejeno, torej obstaja njegova limita $\lim_{n \to \infty}b_n = C\in \R$. Naj bo $\tilde{S}(n)=\sum_{k=1}^{n}a_{\sigma(k)}$ neka delna vsota vrste $\sum_{k=1}^{\infty}a_{\sigma(k)}$. Po enakosti \ref{eq:5} velja: $$\tilde{S}(n)=S(d_1^n)+\sum_{i=2}^{b_n}(S(d_i^n)-S(c_i^n-1))\leq S(d_1^n)+C(S(d_i^n)-S(c_i^n-1)).$$ Vidimo, da je $\tilde{S}(n)$ linearna kombinacija $B=2C+1$ delnih vsot $S(i)$ prvotne vrste.
\end{proof}

\subsection{Množici $\mathcal{O}$ in $\mathcal{N}$}

Naj bo $\mathcal{O}$ množica vseh permutacij naravnih števil, ki ohranjajo konvergenco, vendar niso $\lambda$-permutacije, z $\mathcal{N}$ pa označimo množico vseh $\lambda$-permutacij. Oglejmo si nekaj njunih lastnosti.

\begin{trditev}\label{trd:o in n}
Velja:
\begin{enumerate}
\item če je permutacija $\sigma$ element množice $\mathcal{N}$, potem $\sigma^{-1}$ ni element $\mathcal{O}$ \label{itm1:1}
\item če je permutacija $\sigma$ element množice $\mathcal{N}$  in $\beta$ element množice $\mathcal{O}$, potem je tudi $\sigma \circ \beta$ element $\mathcal{N}$ \label{itm1:2}
\item če je permutacija $\sigma$ element $\mathcal{N}$, potem je tudi $\sigma \circ \sigma$ element $\mathcal{N}$ \label{itm1:3}
\item množica $\mathcal{N}$ je polgrupa za kompozitum permutacij, ni pa grupa \label{itm1:4}
\item permutacija $\sigma$ je element množice $\mathcal{O}$ natanko tedaj ko tako $\sigma$ kot tudi $\sigma^{-1}$ ohranjata konvergenco.
\end{enumerate}
\end{trditev}

\begin{proof}
\begin{enumerate}
\item Naj bo $\sigma$ permutacija iz množice $\mathcal{N}$. Ker $\sigma$ ohranja konvergenco, obstaja neka pogojno divergentna vrsta $\sum_{n=1}^{\infty}a_n$, da je $\sum_{n=1}^{\infty}\sigma(a_n)$ konvergentna. Delujmo sedaj na vrsto  $\sum_{n=1}^{\infty}\sigma(a_n)$ s permutacijo $\sigma^{-1}$. Dobimo vrsto $\sum_{n=1}^{\infty}\sigma^{-1}(\sigma(a_n))=\sum_{n=1}^{\infty}a_n$, ki je divergentna. Permutacija $\sigma^{-1}$ torej ne ohranja konvergence in zato ni element množice $\mathcal{O}$.

\item Naj bo $\sigma \in \mathcal{N}$ in $\beta \in \mathcal{O}$. Tako $\sigma$ kot tudi $\beta$ ohranjata konvergenco, od koder sledi, da tudi njun kompozitum $\sigma \circ \beta$ ohranja konvergenco. Ker je $\sigma$ element množice $\mathcal{N}$, obstaja neka pogojno divergentna vrsta $\sum_{n=1}^{\infty}a_n$, da $\sum_{n=1}^{\infty}\sigma(a_n)$ konvergira. Ker permutacija $\beta$ ohranja konvergenco, je tudi $\sum_{n=1}^{\infty}\beta(\sigma(a_n))=\sum_{n=1}^{\infty}(\beta \circ \sigma)(a_n)$ konvergentna. Torej obstaja pogojno divergentna vrsta, ki jo permutacija $\beta \circ \sigma$ spremeni v konvergentno. Sledi, da je $\beta \circ \sigma \in \mathcal{N}$.

\item Imejmo permutacijo $\sigma \in \mathcal{N}$. Ker $\sigma$ ohranja konvergenco, jo ohranja tudi $\sigma \circ \sigma$. Ker je $\sigma \in \mathcal{N}$, pa obstaja tudi pogojno divergentna vrsta $\sum_{n=1}^{\infty}a_n$, da vrsta $\sum_{n=1}^{\infty}\sigma(a_n)$ konvergira. To isto vrsto $\sum_{n=1}^{\infty}a_n$ tudi $\sigma \circ \sigma$ preslika v konvergentno, saj $\sigma$ ohranja konvergenco. Sledi, da je $\sigma \circ \sigma \in \mathcal{N}$.

\item Najprej dokažimo, da je množica $\mathcal{N}$ polgrupa za kompozitum. Dokazati moramo zaprtost operacije in asociativnost. Asociativnost kompozituma je očitna, zaprtost operacije pa sledi iz točke \ref{itm1:3} v trditvi \ref{trd:o in n}. Pokažimo še, da $\mathcal{N}$ ni grupa za kompozitum. Opazimo, da $\mathcal{N}$ ne vsebuje enote, torej identitete. Identiteta namreč nobene pogojno divergentne vrste ne preslika v pogojno konvergentno. Poleg tega po točki \ref{itm1:1} v trditvi \ref{trd:o in n} množica $\mathcal{N}$ ne vsebuje inverza nobenega svojega elementa, saj za elemente množice $\mathcal{N}$ velja, da njihovi inverzi ne ohranjajo konvergence.
%dokaz za zadnjo točko
\end{enumerate}
\end{proof}

Posledica prejšnje trditve nam pravzaprav podaja alternativno definicijo \\$\lambda$-permutacij.

\begin{posledica}
Permutacija naravnih števil $\sigma$ je $\lambda$-permutacija natanko tedaj ko ohranja konvergenco, njen inverz pa ne.
\end{posledica}
%ali posledica rabi dokaz?

\section{Množica vseh $\lambda$-permutacij}

\subsection{Moč množice $\lambda$ permutacij}

Pokazali smo že, da množica $\lambda$-permutacij ni prazna.
O njeni moči govori naslednji izrek.

\begin{izrek}
$card(\mathcal{N})=card(\mathcal{O})=2^{\aleph_0}$
\end{izrek}

\begin{proof}
Naj bo množica $X$ poljubna podmnožica množice naravnih števil $\N$. Naj bo $\tau_X$ permutacija, ki transponira števili $2m-1$ in $2m$, če je $m\in X$, ostala števila pa preslika sama vase. %Vidimo lahko, da je zaporedje števil blokov $\{b_n\}_{n=1}^{\infty}$ omejeno. Še več, število blokov $b_n$ je za vsak $n$ enako ali $1$ ali $2$. Oglejmo si, zakaj. 

Oglejmo si  primera dveh takih permutaciji $\tau_X$ in $\tau_Y$. Naj bo $X=\{m\}$ in $Y=\{m,m+1\}$. Permutacija $\tau_X$ je videti tako:
$$\tau_X =
\left(
\begin{array}{cccccccc}
1 & 2 & \cdots & 2m-2 & 2m-1 & 2m & 2m+1 & \cdots \\  
1 & 2 & \cdots & 2m-2 & 2m & 2m-1 & 2m+1 & \cdots
  \end{array}
\right)$$
Permutacija $\tau_X$ je torej transpozicija $\tau_X = (2m-1 \: 2m)$.
Potem velja naslednje:
$$\{ \sigma (1), \sigma (2), \ldots \sigma (k) \} = [1, k]_{\Z}$$
za vsak $k\in \N$, različen od $2m-1$ za $m\in X$. Za $k=2m-1$ pa velja:
$$\{ \sigma (1), \sigma (2), \ldots \sigma (k) \} =\{ \sigma (1), \sigma (2), \ldots \sigma (2m-1) \}= [1, 2m-2]_{\Z} \cup [2m, 2m-2]_{\Z}$$
Permutacija $\tau_Y$ pa izgleda tako:
$$\tau_Y =
\left(
\begin{array}{cccccccccc}
1 & 2 & \cdots & 2m-2 & 2m-1 & 2m & 2m+1 & 2m+2 & 2m+3 & \cdots \\  
1 & 2 & \cdots & 2m-2 & 2m & 2m-1 & 2m+2 & 2m+1 & 2m+3 & \cdots
  \end{array}
\right),$$
torej je $\tau_Y$ produkt transpozicij $\tau_Y=(2m-1 \: 2m)\: (2m+1 \: 2m+2)$.
V tem primeru pa je 
$$\{ \sigma (1), \sigma (2), \ldots \sigma (k) \} = [1, k]_{\Z}$$
za vsak $k\in \N$, različen od $2m+1$ in od $2m-1$ za $m\in Y$. Za $k=2m+1$ oziroma $k=2m-1$ pa velja:
$$\{ \sigma (1), \sigma (2), \ldots \sigma (2m+1) \}= [1, 2m]_{\Z} \cup [2m+2, 2m+2]_{\Z}$$
in 
$$\{ \sigma (1), \sigma (2), \ldots \sigma (2m+1) \}= [1, 2m]_{\Z} \cup [2m+2, 2m+2]_{\Z}$$
%ali je smiselno, da se tega lotevam tako?

Iz teh dveh primerov je razvidno, da je tako definirana permutacija $\tau_X$ vedno enaka produktu transpozicij oblike $(2m-1 \: 2m)$ za $m \in X$ in da za njeno zaporedje števil blokov $\{b_n\}_{n=1}^{\infty}$ zato velja:
$$b_n = 
\left\{ 
\begin{array}{ccc}
2&;&n=2m-1\textrm{ za nek }m \in X\\
1&;&\textrm{sicer.}\\
\end{array}
\right. 
$$

Njeno zaporedje števil blokov je torej omejeno, od koder po trditvi \ref{trd:ohranjanje} sledi, da permutacija $\tau_X$ na številskih vrstah ohranja konvergenco in vsoto. Torej je $\tau_X \in \mathcal{O}$.

Za različne množice $X \subseteq \N$ dobimo različne permutacije $\tau_X$. Ker je podmnožic množice naravnih števil ravno $2^{\aleph_0}$, imamo tudi $2^{\aleph_0}$ permutacij oblike $\tau_X$. Sledi, da je $card(\mathcal{O})\geq2^{\aleph_0}$. Po drugi strani pa vemo, da ima množica vseh permutacij naravnih števil moč $2^{\aleph_0}$, torej lahko velja le $card(\mathcal{O})\leq2^{\aleph_0}$. Sledi, da je $card(\mathcal{O})=2^{\aleph_0}$.

Oglejmo si še, kako je z močjo množice $\lambda$-permutacij oziroma množice $\mathcal{N}$. Iz trditve \ref{trd:o in n} je razvidno, da je kompozicija permutacije iz množice $\mathcal{O}$ z $\lambda$-permutacijo spet $\lambda$-permutacija. Naj bo $\sigma$ neka $\lambda$-permutacija (v poglavju \ref{sec:konstrukcija} smo pokazali, da obstaja vsaj ena $\lambda$-permutacija) in naj bo $\tau_X$ prej definirana permutacija, ki je, kot smo pokazali, element množice $\mathcal{O}$. Torej je kompozitum obeh permutacij $\tau_X \circ \sigma$ spet $\lambda$-permutacija. Enako kot pri računanju moči množice $\mathcal{O}$, tudi tu opazimo, da mora biti moč množice $\lambda$-permutacij vsaj $2^{\aleph_0}$, saj je toliko vseh množic $X \subseteq \N$, hkrati pa ne more biti večja od moči množice vseh permutacij naravnih števil. Tako da je tudi  $card(\mathcal{N})=2^{\aleph_0}$.
\end{proof}

\subsection{Omejenost množice $\lambda$ permutacij}

Ali ima množica vseh $\lambda$-permutacij zgornjo mejo, tj.\ ali obstaja permutacija, ki ustvari največ konvergentnih vrst?

Da bomo lahko odgovorili na to vprašanje, najprej definirajmo relacijo delne urejenosti na množici $ \mathbb{N}$. Naj bosta $\sigma _1$ in $\sigma _2$ $\lambda$-permutaciji. Rečemo, da je $\sigma_1 < \sigma_2$, če za vsako vrsto $\sum^{\infty}_{n=1}a_n$, za katero $\sum^{\infty}_{n=1}a_{\sigma_2(n)}$ konvergira, konvergira tudi $\sum^{\infty}_{n=1}a_{\sigma_1(n)}$. 

Pokazali smo že (trditev \ref{trd:o in n}), da če je $\sigma$ $\lambda$-permutacija, potem je tudi $\sigma \circ \sigma$  $\lambda$-permutacija. Če vrsta $\sum^{\infty}_{n=1}a_{\sigma(n)}$ konvergira, torej konvergira tudi $\sum^{\infty}_{n=1}a_{\sigma \circ \sigma(n)}$. Velja torej $\sigma < \sigma \circ \sigma$, od tod pa induktivno sledi: $\sigma < \sigma \circ \sigma < \sigma \circ \sigma \circ \sigma < \cdots$. Za vsako $\lambda$-permutacijo lahko najdemo novo $\lambda$-permutacijo, ki bo ">večja"< od nje. Torej množica vseh $\lambda$-permutacij nima zgornje meje. 
%tukaj nekje pride topologija?


\section{Pogojno divergentne vrste in $\lambda$-permutacije}

V tem razdelku bomo poskušali odgovoriti še na vprašanje, ali za vsako pogojno divergentno vrsto obstaja $\lambda$-permutacija, taka, da vrsta s spremenjenim vrstnim redom členov konvergira. V ta namen za vsako pogojno konvergento oziroma pogojno divergentno vrsto definirajmo množico $$S=\left \{L\in \mathbb{R}:\textrm{obstaja }\lambda \textrm{-permutacija } \sigma, \sum^{\infty}_{n=1}a_{\sigma (n)}=L\right \}.$$ Na vprašanje bomo odgovorili s pomočjo protiprimera, torej primera pogojno divergentne številske vrste, za katero je $S=\emptyset$. %v drugem delu tega razdelka si bomo ogledali še blablabla

Oglejmo si naslednjo številsko vrsto:
\begin{equation}\label{eq:6}
\sum^{\infty}_{n=1}a_n = 1+\frac{1}{2}+\frac{1}{3}+\frac{1}{4}-1+\frac{1}{5}+\frac{1}{6}+ \dots +\frac{1}{33}-\frac{1}{2}+\dots
\end{equation}
Zaporedje $a_n$, ki ga seštevamo, je zgrajeno iz med sabo prepletenih zaporedij $\{\frac{1}{n}\}_{n=1}^{\infty}$ in $\{-\frac{1}{n}\}_{n=1}^{\infty}$. Začne se s prvimi nekaj členi zaporedja $\{\frac{1}{n}\}_{n=1}^{\infty}$. Teh mora biti toliko, da je njihova vsota večja ali enaka $2$. Sledi prvi element zaporedja $\{-\frac{1}{n}\}_{n=1}^{\infty}$, nato pa naslednjih nekaj členov zaporedja $\{\frac{1}{n}\}_{n=1}^{\infty}$, spet toliko, da njihova vsota doseže ali preseže $2$, potem pa drugi elelemnt zaporedja $\{-\frac{1}{n}\}_{n=1}^{\infty}$ in tako dalje. 
Vrsta je pogojno divergentna. Da je divergentna vidimo, če si ogledamo njeno zaporedje delnih vsot. %ne znam razložiti naprej
Da je pogojno divergentna, pa sledi iz dejstva, da obstaja pogojno konvergentna vrsta $$\sum_{n=1}^{\infty}\left(\frac{1}{n}-\frac{1}{n}\right),$$ %ali je to pravilno %zapiši raje po členih
sestavljena iz popolnoma enakih členov kot vrsta (\ref{eq:6}). Ta konvergira po testu za alternirajoče vrste. %sklicevanje na ta test?
Torej mora obstajati nekakšna permutacija, ki preuredi člene vrste (\ref{eq:6}), da dobimo pogojno konvergentno vrsto.
Trdimo, da tega ni mogoče storiti z nobeno $\lambda$-permutacijo, torej da za je to vrsto $S=\emptyset$. 
%ali bi morala imeti tu strukturo trditev-dokaz?

Zakaj je temu tako, bomo pokazali s protislovjem. Naj bo $\sigma$ neka poljubna $\lambda$-permutacija, s katero delujemo na vrsto (\ref{eq:6}), in naj bo zaporedje števil blokov $\{b_n\}_{n=1}^{\infty}$ omejeno z $B$. Če delne vsote vrste $\sum_{n=1}^{\infty}a_{\sigma(n)}$ zapišemo kot v enakosti (\ref{eq:5}), dobimo:
$$\sum_{k=1}^{n}a_{\sigma(k)}=\sum_{k=1}^{d_1^n}a_{k} + \sum_{i=2}^{b_n}\sum_{k=c_i^n}^{d_i^n}a_{k} \geq \sum_{k=1}^{d_1^n}a_{k} - (b_n -1) \geq \sum_{k=1}^{d_1^n}a_{k} - (B -1).$$
Očitno je namreč, da vsota katerihkoli zaporednih členov zgoraj definiranega zaporedja $\{a_n\}_{n=1}^{\infty}$ nikoli ni manjša od $-1$. Torej velja, da je $$\sum_{k=c_i^n}^{d_i^n}a_{k} \geq -1.$$ Sledi:
$$\sum_{i=2}^{b_n}\sum_{k=c_i^n}^{d_i^n}a_{k} \geq \sum_{i=2}^{b_n}(-1) = -(b_n-1).$$
Ko pošljemo $n$ proti neskončno, gre tudi $d_1^n$ proti neskončno (kar smo videli že v dokazu izreka ??). Vrste oblike $\sum_{k=1}^{d_1^n}a_{k}$ so delne vsote vrste (\ref{eq:6}), katerih zaporedje gre proti neskončno, ko gre $d_1^n$ proti neskončno, saj vrsta (\ref{eq:6}) divergira. Sledi, da tudi zaporedje delnih vsot $\sum_{k=1}^{n}a_{\sigma(k)}$ divergira. Ker je bila izbira permutacije $\sigma$ poljubna, lahko zaključimo, da vrsta $\sum_{n=1}^{\infty}a_{\sigma(n)}$ ne konvergira za nobeno $\lambda$-permutacijo $\sigma$.


Izrek \ref{izr:riemann} pravi, da je v vsaki pogojno konvergentni vrsti mogoče spremeniti vrstni red členov tako, da dobljena vrsta divergira ali pa konvergira proti kateremukoli realnemu številu. Zgolj z uporabo $\lambda$-permutacij lahko to dosežemo pri naslednji vrsti:
\begin{equation}\label{eq:7}
\sum^{\infty}_{n=1}a_n= 1-1-\frac{1}{2}-\frac{1}{3}-\frac{1}{4}+\frac{1}{2}+\frac{1}{3}+\dots +\frac{1}{33}-\frac{1}{5}-\dots
\end{equation}
Zaporedje $\{a_n\}_{n=1}^{\infty}$ je tudi v tem primeru zgrajeno iz med sabo prepletenih zaporedij $\{\frac{1}{n}\}_{n=1}^{\infty}$ in $\{-\frac{1}{n}\}_{n=1}^{\infty}$. Začne se s prvimi nekaj členi zaporedja $\{\frac{1}{n}\}_{n=1}^{\infty}$, takimi, da je njihova vsota enaka $s_1$. Tu je $s_1$ najmanjše število, da velja $s_1 \geq 1$. Sledi prvih nekaj števil zaporedja $\{-\frac{1}{n}\}_{n=1}^{\infty}$, takih, da je njihova vsota enaka $-t_1$, pri čemer je $t_1$ najmanjše možno število, za katero velja $t_1 \geq s_1+1$. Nadaljujemo z elementi zaporedja $\{\frac{1}{n}\}_{n=1}^{\infty}$, ki jih je toliko, da je njihova vsota enaka $s_2$, pri čemer je $s_2\geq t_1+1$, in tako dalje. %primer?

Za lažjo obravnavo pozitivne člene zaporedja $a_n$, katerih vsota je $s_i$ imenujmo \emph{$i$-ti pozitivni blok}, negativne člene, katerih vsota je $-t_i$ pa \emph{$i$-ti negativni blok},

Da je ta vrsta pogojno konvergentna, vidimo podobno kot pri prejšnjem primeru. %referenca? %razloži  

Trdimo, da za vsako število $L\in \R$ obstaja $\lambda$-permutacija, ki iz zgornje (divergentne) vrste (\ref{eq:7}) ustvari konvergentno vrsto z vsoto $L$, torej da je za to vrsto $S=\R$. %malo več nakladanja %naj bo to samostojna trditev?

Naj bo torej $L$ neko realno število. Poiskali bomo $\lambda$-permutacijo $\sigma$, ki bo ustrezno preoblikovala vrsto (\ref{eq:7}). 

Najprej poiščimo tak indeks $m_1$, da bo za prej definirano zaporedje $\{a_n\}_{n=1}^{\infty}$ veljalo $$a_{m_1}>0\textrm{ in }L>\sum_{k=1}^{m_1}a_k \geq L+1.$$ Tak indeks $m_1$ zagotovo obstaja. Iz konstrukcije zaporedja $\{a_n\}_{n=1}^{\infty}$ namreč sledi, da delne vsote vrste (\ref{eq:7}) stalno nihajo z vedno večjo frekvenco med pozitivnimi in negativnimi celimi števili. Za prvo delno vsoto te vrste na primer velja $S(1)\geq 1$, za drugo velja, da je $S(2)\leq -1$, sledi $S(3) \geq 2$, $S(4) \leq -2$, in tako dalje.

Definirajmo permutacijo $\sigma$ na prvih $m_1$ elementih naravnih števil kot identiteto, torej $$\sigma(k)=k\textrm{ za }k\leq m_1.$$ Predpostavimo sedaj, da je $a_{m_1}$ del $j$-tega pozitivnega bloka, torej eden izmed členov zaporedja $\{a_n\}_{n=1}^{\infty}$, katerih vsota je $s_j$. Člen $a_{m_1}$ je namreč pozitivno število. Nadaljujmo s konstrukcijo permutacije $\sigma$. Vrednost $\sigma(m_1+1)$ naj bo taka, da je člen $a_{\sigma(m_1+1)}$ enak prvemu še ne uporabljenemu (v tem primeru kar prvemu) členu zaporedja $\{a_n\}_{n=1}^{\infty}$, ki pripada $j$-temu negativnemu bloku, torej bloku števil z vsoto $-t_j$. 
%primer?
Preostanek permutacije $\sigma$ skonstruiramo po podobnem principu. Recimo, da smo do sedaj določili $m_i$ členov zaporedja $\{a_\sigma(n)\}_{n=1}^{\infty}$. Če za njihovo vsoto velja $$\sum^{m_i}_{n=1}a_\sigma(n) \leq L,$$ bomo vrednost $\sigma(m_i+1)$ določili tako, da bo $a_{\sigma(m_1+1)}$ enak prvemu še neuporabljenemu pozitivnemu členu zaporedja $\{a_n\}_{n=1}^{\infty}$. Če pa je $$\sum^{m_i}_{n=1}a_\sigma(n) > L,$$ bo vrednost $\sigma(m_i+1)$ taka, da bo člen $a_{\sigma(m_1+1)}$ enak prvemu še ne uporabljenemu negativnemu členu zaporedja $\{a_n\}_{n=1}^{\infty}$.%primer?

Vrsta $\sum^{\infty}_{n=1}a_{\sigma(n)}$, ki jo dobimo, ko končamo s konstrukcijo permutacije $\sigma$ ima vsoto $L$. Permutacijo $\sigma$ smo namreč skonstruirali tako, da delne vsote $\tilde{S}(n)$ vrste $\sum^{\infty}_{n=1}a_{\sigma(n)}$ stalno nihajo okrog $L$. Dve zaporedni delni vsoti $\tilde{S}(k)$ in $\tilde{S}(k)$ se torej nahajata na intervalu $(L-\epsilon, L+\epsilon)$ za nek $\epsilon > 0$. Poleg tega pa se, ko raste $n$, členi $a_{\sigma(n)}$ manjšajo, torej je razlika med dvema zaporednima delnima vsotama vedno manjša. Interval, na katerem se nahajata zaporedni delni vsoti, se torej manjša. Z drugimi besedami: ko gre $n$ proti neskončnosti, gre $\epsilon$ proti $0$ in posledično vsota vrste $\sum^{\infty}_{n=1}a_{\sigma(n)}$ proti $L$.

Opazimo pa, da velja tudi $$L-1<\sum_{k=1}^{n}a_{\sigma(k)} \geq L+1\textrm{ za }n\geq m_1.$$ Členi zaporedja $\{a_n\}_{n=1}^{\infty}$ namreč vsi ležijo na intervalu $[-1,1]$, zato tudi $\epsilon$ v intervalu $(L-\epsilon, L+\epsilon)$, na katerem ležita zaporedni delni vsoti vrste $\sum_{k=1}^{\infty}a_{\sigma(k)}$, nikoli ni večji od $1$. Vsaka delna vsota $\sum_{k=1}^{n}a_{\sigma(k)}$ od $m_1$-te dalje torej leži na nekem podintervalu intervala $[L-1,L+1]$. %kako je z oklepaji (oglati, okrogli?)

Sedaj moramo dokazati še, da je dobljena permutacija $\sigma$ $\lambda$-permutacija. Omenimo najprej nekaj koristnih opažanj. Vidimo lahko, da je $a_{\sigma(m_1+1)}$ enak ravno prvemu členu $(j+1)$-tega negativnega bloka. Ker je za $k \leq m_1$ $\sigma(k)=k$ in ker je $\sigma(m_1)$ element $j$-tega pozitivnega bloka, pred $\sigma(m_1)$ v zaporedju $\{a_{\sigma(n)}\}_{n=1}^{\infty}$ ni nastopilo nobeno število iz $(j+1)$-tega negativnega bloka. Torej je v tem primeru prvi člen $(j+1)$-tega negativnega bloka ravno prvi še neuporabljeni člen tega bloka. Poleg tega vidimo tudi, da prvih $m_1$ členov vrste $\sum_{k=1}^{\infty}a_{\sigma(k)}$ vsebuje vse člene prvotne vrste do vključno $(j-1)$-tega bloka (in morda še kakšnega iz $j$-tega pozitivnega bloka.

Trdimo sedaj, da se v vrsti  $\sum_{k=1}^{\infty}a_{\sigma(k)}$ prvi člen $(j+1)$-tega pozitivnega bloka pojavi pred prvim členom $(j+1)$-tega negativnega bloka. %povej, zakaj moramo to videti
Dokaza se lotimo s protislovjem. Recimo, da se prvi člen $(j+1)$-tega negativnega bloka pred prvim členom $(j+1)$-tega pozitivnega bloka in naj se pojavi na $m'+1$-tem mestu permutirane vrste $\sum_{k=1}^{\infty}a_{\sigma(k)}$. Potem mora biti za vsako vrednost $k$, ki zadošča pogoju $m_1<k\leq m'$, $k$-ti člen permutirane vrste ali iz $j$-tega pozitivnega bloka ali iz $j$-tega negativnega bloka. Poleg tega morajo za tak $k$ med člene oblike $a_\sigma(k)$ spadati ravno vsi elementi $j$-tega negativnega bloka, saj lahko pri konstrukciji permutacije $\sigma$ oziroma zaporedja $\{a_{\sigma(n)}\}_{n=1}^{\infty}$ začnemo uporabljati elemente $(j+1)$-tega negativnega bloka šele po tem, ko uporabimo vse elemente $j$-tega negativnega bloka. Vidimo tudi, da mora biti $$\sum_{k=1}^{m'}a_{\sigma(k)}>L,$$saj je šele takrat, ko je delna vsota permutirane vrste večja od $L$, naslednji člen v permutirani vrsti lahko negativen.
Vidimo tudi, da vrsta $\sum_{k=1}^{m'}a_{\sigma(k)}$ zajema člene iz vseh blokov do vključno $j$-tega negativnega bloka. %ali je jasno zakaj?
Vrsta $\sum_{k=1}^{m_1}a_{\sigma(k)}$ pa zagotovo zajema vse člene do vključno $(j-1)$-tega negativnega bloka in morda še kakšen člen iz $j$-tega bloka. Zato je vsota $$\sum_{k=1}^{m_1}a_{\sigma(k)}+s_j-t_j,$$kjer je $s_j$ vsota $j$-tega pozitivnega bloka in $-t_j$ vsota $j$-tega negativnega bloka, večja ali enaka vsoti $\sum_{k=1}^{m'}a_{\sigma(k)}$. Iz konstrukcije permutacije $\sigma$ sledi, da je $$\sum_{k=1}^{m_1}a_{\sigma(k)}\leq L+1,$$ iz lastnosti vrste (\ref{eq:7}) pa, da je $s_j-t_j \leq -1$. Imamo torej: $$L< \sum_{k=1}^{m'}a_{\sigma(k)}\leq  \sum_{k=1}^{m_1}a_{\sigma(k)}+s_j-t_j \leq L+1-1=L.$$
Pridemo v protislovje.

Pokazali smo torej, da se prvi člen $(j+1)$-tega pozitivnega bloka v permutirani vrsti pojavi pred prvim členom $(j+1)$-tega negativnega bloka. Recimo, da se pojavi na $(m_2+1)$-tem mestu. Velja, da prvih $m_2$ členov permutirane vrste vključuje vse člene originalne vrste do vključno $j$-tega pozitivnega bloka in nekaj členov $j$-tega negativnega bloka. %malo bolj jasno tisto kar bo sledilo........................situacija podobna kot za m_1
Na podoben način kot prej vidimo, da je prvi člen $(j+1)$-tega negativnega bloka uporabljen pred prvim členom $(j+2)$-tega pozitivnega bloka. %kako to? kaj pa če bi kar razširila v svoj dokaz?
Po indukciji sledi, da je prvi člen vsakega bloka uporabljen pred prvim členom naslednjega bloka.

Od tod vidimo, da je zaporedje števil blokov $\{b_n\}_{n=1}^{\infty}$ za permutacijo $\sigma$ omejeno z $2$. %kako naj to razložim help!
Po izreku (ref ??) sledi, da permutacija $\sigma$ ohranja konvergenco, torej ustreza prvi točki iz definicije $\lambda$-permutacij. Ker smo permutacijo $\sigma$ zgradili tako, da pogojno divergentno vrsto preoblikuje v pogojno konvergentno, torej ustreza tudi drugi točki iz definicije $\lambda$-permutacij. Permutacija $\sigma$ je torej $\lambda$-permutacija.

Ogledali smo si taka primera pogojno divergentnih vrst, da je za eno veljalo $S=\emptyset$, za drugo pa $S=\R$. Pojavi se vprašanje, ki naj ostane neodgovorjeno: Je lahko $S$ še kaj drugega kot le $\R$ in $\emptyset$.

\section{Še dva primera...}

Za konec si oglejmo še dva primera $\lambda$-permutacij.

\begin{primer}
Razdelimo naravna števila v bloke na naslednji način:\\

$1 \quad 2\: 3\: 4\: 5 \quad 6\: 7\: 8\: 9\: 10\: 11 \quad 12\: 13\: 14\: 15\: 16\: 17\: 18\: 19\: \ldots $\\

V bloke smo jih razdelili tako, da ima vsak blok razen prvega dolžino $2n$, kjer je $n$ zaporedna številka bloka. Drugi blok ima torej dolžino $4$, tretji blok ima dolžino $6$ in tako dalje.

Zaradi lažje obravnave poimenujmo bloke tako:
\begin{eqnarray*}
B_1&=& 1\\
B_2&=& 2\: 3\: 4\: 5\\
B_3&=& 6\: 7\: 8\: 9\: 10\: 11\\
&\vdots &
\end{eqnarray*}

Sedaj vsak blok $B_i$ razdelimo na dva enako velika boka $B_i^1$ in $B_i^2$ in ju prepletemo med sabo. Blok $B_3=6\: 7\: 8\: 9\: 10\: 11$ na primer razdelimo na bloka $B_3^1=6\: 7\: 8$ in $B_3^2=9\: 10\: 11$ ter ju prepletemo tako, da na prvo mesto postavimo prvo število bloka $B_3^1$, torej $6$, na drugo mesto prvo število bloka $B_3^2$, torej $9$, na tretje mesto drugo število bloka $B_3^1$ in tako dalje. Ko opisani postopek izvedemo na vseh blokih v zaporedju, dobimo:\\ 

$1 \quad 2\: 4\: 3\: 5 \quad 6\: 9\: 7\: 10\: 8\: 11 \quad 12\: 16\: 13\: 17\: 14\: 18\: 15\: 19\: \ldots $\\

Oziroma:
\begin{eqnarray*}
\tilde{B}_1&=& 1\\
\tilde{B}_2&=& 2\: 4\: 3\: 5\\
\tilde{B}_3&=& 6\: 9\: 7\: 10\: 8\: 11\\
&\vdots &
\end{eqnarray*}

Pokažimo, da je permutacija (označimo jo s $\sigma_1$), ki jo implicira zgornja preureditev števil, $\lambda$-permutacija.

Najprej si bomo ogledali zaporedje števil blokov $\{b_n\}_{n=1}^{\infty}$ permutacije $\sigma_1$ in pokazali, da je omejeno. Če je omejeno, namreč permutacija $\sigma_1$ po izreku (ref ??) ohranja konvergenco, torej zadošča prvi točki iz definicije $\lambda$-permutacij. 

Naj bo $B_i = m,\ m+1, \dots, m+n-1,\ m+n$ nek poljuben blok. Na tem bloku uporabimo permutacijo $\sigma_1$. Razdelimo ga na bloka $B_i^1= m,\ m+1, \dots,\ m+\frac{n}{2}-1$ in $B_i^2= m+\frac{n}{2},\ m+\frac{n}{2}+1, \dots,\ m+n$, ki ju prepletemo med sabo in dobimo blok $\tilde{B}_i =  m,\ m+\frac{n}{2}, \ m+1, \dots,\ m+n-1, \ m+\frac{n}{2}-1,\ m+n$. Opazimo lahko, da se blok $\tilde{B}_i$ zaradi načina kako prepletemo bloka $B_i^1$ in $B_i^2$ začne z istim številom kot blok $B_i$ in konča prav tako z istim številom kot $B_i$, torej velja $\sigma_1(m)=m$ in $\sigma_1(m+n)=m+n$. Ker permutiranje oziroma sprememba vrstnega reda števil poteka samo znotraj posameznega bloka in ne med različnimi bloki, blok $\tilde{B_i}$ vsebuje enaka števila kot $B_i$, torej ravno vsa naravna števila med $m$ in $m+n$, le da v drugačnem vrstnem redu. Zato velja $$\{\sigma_1(m), \sigma_1(m+1), \ldots \sigma_1(m+n) \}=[m,m+n]_{\Z}.$$
Naj bosta sedaj $B_i = m,\ m+1, \dots, m+n-1,\ m+n$ in $B_{i-1} = m-k,\ m-k+1, \dots, m-2,\ m-1$ zaporedna bloka. Potem je 
\begin{align*}
&\{\sigma_1(m-k), \sigma_1(m-k+1), \ldots \sigma_1(m+n) \}=\\
=&\{\sigma_1(m-k), \sigma_1(m-k+1), \ldots \sigma_1(m-1) \} \cup \{\sigma_1(m), \sigma_1(m+1), \ldots \sigma_1(m+n) \}=\\
=&[m-k,m-1]_{\Z} \cup [m,m+n]_{\Z}=[m-k]_{\Z}.
\end{align*}
Od tod lahko induktivno sklepamo, da je $$\{ \sigma_1(1), \sigma_1(2), \ldots, \sigma_1(m+n) \}=[1,m+n]_{\Z},$$
kjer je $m+n$ število, s katerim se konča nek blok $B_i$. Za tako število je torej $m+n$-ti element zaporedja števil blokov enak $b_{m+n}=1$.

Podobno vidimo, da je tudi za število $m$, s katerim se začne nek blok $B_i$, $m$-ti element zaporedja števil blokov enak $b_m=1$. S številom $m-1$ se namreč konča blon $B_{i-1}$, torej od prej sledi $$\{ \sigma_1(1), \sigma_1(2), \ldots, \sigma_1(m-1) \}=[1,m-1]_{\Z}.$$
Torej je  $$\{ \sigma_1(1), \sigma_1(2), \ldots, \sigma_1(m) \}=[1,m-1]_{\Z} \cup \{\sigma(m)\}= [1,m-1]_{\Z} \cup \{m\}=[1,m]_{\Z}.$$

Kako pa je z ostalimi elementi zaporedja števil blokov? Naj bo $m+k$ število, ki ni ne na začetku in ne na koncu bloka $B_i$ in naj bo $m+k \in [m,m+n]_{\Z}$ oziroma $\sigma_1(m+k) \in \{\sigma_1(m), \sigma_1(m+1), \ldots, \sigma_1(m+n)$. Iščemo $b_{m+k}$, ki je število blokov oblike $[c,d]_{\Z}$, unija katerih je enaka množici $\{ \sigma_1(1), \sigma_1(2), \ldots, \sigma_1(m+k) \}$. 
Od prej vemo, da je $$\{ \sigma_1(1), \sigma_1(2), \ldots, \sigma_1(m+k) \}=[1,m-1]_{\Z}\cup \{ \sigma_1(m), \sigma_1(m+1), \ldots, \sigma_1(m+k) \}.$$
Oglejmo si množico $\{ \sigma_1(m), \sigma_1(m+1), \ldots, \sigma_1(m+k) \}.$ Vsebuje števila iz bloka $\tilde{B}_i$, torej se v njej izmenjujejo števila iz blokov $B_i^1$ in $B_i^2$. Števila iz vsakega izmed obeh blokov si sledijo po vrsti, torej, če sta števili $n_1$ in $n_2$ obe vsebovani na primer v bloku $B_i^1$ in je $n_1<n_2$, bo tudi v permutiranem bloku $\tilde{B}_i$ število $n_1$ nastopilo pred številom $n_2$. Enako velja, če sta obe vsebovani v bloku $B_i^2$.
Sledi, da se množica vseh števil iz bloka $B_i^1$, ki so vsebovana v množici $\{ \sigma_1(m), \sigma_1(m+1), \ldots, \sigma_1(m+k) \},$ da zapisati kot $[m,m+j]$, kjer je $$m=\min \{B_i^1 \cap \{ \sigma_1(m), \sigma_1(m+1), \ldots, \sigma_1(m+k) \} \}$$
in $$m+j=\max \{B_i^1 \cap \{ \sigma_1(m), \sigma_1(m+1), \ldots, \sigma_1(m+k) \} \}$$
in mnoica vseh števil iz bloka $B_i^2$, ki so vsebovana v množici $\{ \sigma_1(m), \sigma_1(m+1), \ldots, \sigma_1(m+k) \},$ da zapisati kot $[m+o,m+p]$, kjer je $$m+o=\min \{B_i^2 \cap \{ \sigma_1(m), \sigma_1(m+1), \ldots, \sigma_1(m+k) \} \}$$
in $$m+p=\max \{B_i^2 \cap \{ \sigma_1(m), \sigma_1(m+1), \ldots, \sigma_1(m+k) \} \}.$$
Sledi $$\{ \sigma_1(m), \sigma_1(m+1), \ldots, \sigma_1(m+k) \}=[m,m+j]_{\Z} \cup [m+o,m+p]_{\Z}.$$
Dobimo torej, da je $$\{ \sigma_1(1), \sigma_1(m+1), \ldots, \sigma_1(m+k) \}= [1,m-1]_{\Z} \cup [m,m+j]_{\Z} \cup [m+o,m+p]_{\Z} = [1,m+j]_{\Z} \cup [m+o,m+p]_{\Z}.$$
Če je $m+j+1<m+o$, je dobljeni zapis končen in za $m+k$-ti element zaporedja števil blokov velja $b_{m+k}=2$. Če pa je $m+j+1=m+o$, se da zgornji zapis poenostaviti na $$\{ \sigma_1(1), \sigma_1(m+1), \ldots, \sigma_1(m+k) \}=[1,m+k]_{\Z}$$
in v tem primeru je $b_{m+k}=1$.
Ker je bila izbira števila $m+k$ poljubna, sledi, da so vsi členi zaporedja števil blokov za permutacijo $\sigma_1$ enaki ali $1$ ali $2$, torej je zaporedje števil blokov omejeno in po izreku (ref??) permutacija $\sigma_1$ ohranja konvergenco.

Da pokažemo, da je $\sigma_1$ $\lambda$-permutacija, moramo dokazati še, da zadošča drugi točki iz definicije $\lambda$-permutacij, torej da obstaja neka pogojno divergentna vrsta, ki ji $\sigma_1$ preuredi člene tako, da dobimo pogojno konvergentno vrsto. Primer take pogojno konvergentne vrste je naslednji:
\begin{align}\label{eq:8}
\sum_{n=1}^{\infty}a_n = & 1+\left(-\frac{1}{2}-\frac{1}{2}+\frac{1}{2}+\frac{1}{2}\right)+\left(-\frac{1}{3}-\frac{1}{3}-\frac{1}{3}+\frac{1}{3}+\frac{1}{3}+\frac{1}{3}\right)+ \nonumber \\ 
&+\left(-\frac{1}{4}-\frac{1}{4}-\frac{1}{4}-\frac{1}{4}+\frac{1}{4}+\frac{1}{4}+\frac{1}{4}+\frac{1}{4}\right)+\cdots
\end{align}
Zaporedje delnih vsot te vrste je videti tako: $$S_1=1,\ S_2=-\frac{1}{2}, \ S_3=0, \ S_4=\frac{1}{2}, \ S_5=1, \ S_6=\frac{2}{3}, \ S_7=\frac{1}{3}, \ S_8=0, \ \ldots$$
Vidimo, da to zaporedje delnih vsot med drugim vsebuje podzaporedje, sestavljeno iz samih enk, in podzaporedje sestavljeno iz samih ničel. Zaporedje ima torej dve različni stekališči, $1$ in $0$, od koder sledi, da ni konvergentno.

Če sedaj člene vrste \eqref{eq:8} preuredimo s permutacijo $\sigma_1$, dobimo vrsto 
\begin{align}\label{eq:9}
\sum_{n=1}^{\infty}a_{\sigma(n)} = & 1-\frac{1}{2}+\frac{1}{2}-\frac{1}{2}+\frac{1}{2}-\frac{1}{3}+\frac{1}{3}-\frac{1}{3}+\frac{1}{3}-\frac{1}{3}+\frac{1}{3}- \nonumber \\ 
&-\frac{1}{4}+\frac{1}{4}-\frac{1}{4}+\frac{1}{4}-\frac{1}{4}+\frac{1}{4}-\frac{1}{4}+\frac{1}{4}\pm\cdots
\end{align}
Ta pa konvergira po konvergenčnem testu za alternirajoče vrste. %ali moram povedati zakaj? 

Permutacija $\sigma_1$ je torej res $\lambda$-permutacija.
\end{primer}

\begin{primer}
Oglejmo si še en primer $\lambda$-permutacije. Zopet začnemo z zaporedjem naravnih števil, razdeljenim v bloke naraščajoče dolžine:\\
$1 \quad 2\: 3\: 4\quad 5\: 6\: 7\: 8\: 9\: 10 \quad 11\: 12\: 13\: 14\: 15\: 16\:17\: 18\: 19\: \ldots $\\
V bloke smo ga razdelili tako, da je njihova dolžina (razen prvega bloka, ki ima dolžino $1$) večkratnik števila $3$. Drugi blok ima na primer dolžino $3$, tretji blok ima dolžino $6$, $n$-ti pa dolžino $3(n-1)$. 

Zaradi lažje obravnave spet poimenujmo bloke tako:
\begin{eqnarray*}
C_1&=& 1\\
C_2&=& 2\: 3\: 4\\
C_3&=& 5\: 6\: 7\: 8\: 9\: 10\\
&\vdots &
\end{eqnarray*}

Ker je dolžina blokov večkratnik števila $3$, lahko vsak blok $C_i$ razdelimo na bloka $C_i^1$ in $C_i^2$, kjer je $C_i^1$ dolg $\frac{2}{3}$ dolžine celotnega bloka $C_i$, blok $C_i^2$ pa $\frac{1}{3}$ dolžine celotnega bloka. Dobljena bloka nato prepletemo med sabo tako, da blok $C_i^1$ razdelimo na podbloke dožine $2$ in jih prepletemo z elementi bloka $C_i^2$. 
Pri bloku $C_3$, na primer, je postopek tak: ker ima blok dolžino $6$, ga razdelimo na blok dolžine $\frac{2}{3} 6=4$ in blok dolžine $\frac{1}{3} 6=2$. Prvi štirje elementi $C_3$ torej prioadajo bloku $C_3^1$, zadnja dva pa $C_3^2$. Blok $C_3^1 = 5\ 6\ 7\ 8$ sedaj razdelimo na podbloka dolžine $2$, torej na $5\ 6$ in $7\ 8$ in ju prepletemo z elementi bloka $C_3^2$, tako da prvi podblok $5\ 6$ postavimo na prvo mesto, sledi prvi element bloka $C_3^2$, potem drugi podblok $7\ 8$ in nato drugi element bloka $C_3^2$. Ta postopek permutiranja poteka nekoliko drugače le pri bloku $C_2$, kjer elementov bloka $C_2^1$ ne moremo deliti na podbloke temveč ju zgolj prepletemo z elementom bloka $C_2^2$.

Na tak način dobimo:

$1 \quad 2\: 4\: 3\quad 5\: 6\: 9\: 7\: 8\: 10 \quad 11\: 12\: 17\: 13\: 14\: 18\: 15\: 16\: 19\: \ldots $

Oziroma:
\begin{eqnarray*}
\tilde{C}_1&=& 1\\
\tilde{C}_2&=& 2\: 4\: 3\\
\tilde{C}_3&=& 5\: 6\: 9\: 7\: 8\: 10\\
&\vdots &
\end{eqnarray*}

Dobljena preureditev števil implicira permutacijo $\sigma_2$, za katero trdimo, da je $\lambda$-permutacija. Za dokaz bomo spet najprej pokazali, da je njeno zaporedje števil blokov $\{b_n\}_{n=1}^{\infty}$ omejeno, nato pa poiskali pogojno divergentno vrsto, ki jo bo $\sigma_2$ preuredila tako, da bo konvergirala.

Opazimo, da za bloke $C_i$, kjer je $i>2$ (permutiranje pri bloku $C_2$ poteka nekoliko drugače kot pri ostalih), velja za prvi dve števili in za zadnje število v bloku, da jih $\sigma_2$ preslika same vase. Naj bo spet $C_i = m,\ m+1, \dots, m+n-1,\ m+n$ nek poljuben blok. Ker zamenjava členov poteka samo znotraj bloka in ne med različnimi bloki, je $$\{\sigma_2(m), \sigma_2(m+1), \ldots, \sigma_2(m+n) \}=[m,m+n]_{\Z}.$$ Enako kot v primeru (ref??)od tod sledi $$\{\sigma_2(1), \sigma_2(2), \ldots, \sigma_2(m+n) \}=[1,m+n]_{\Z}$$
in je $m+n$-ti element zaporedja števil blokov $b_{m+n}=1$. Opazimo, da enako velja tudi za blok $C_2$, čeprav se njegovo zadnje število s permutacijo $\sigma_2$ ne preslika samo vase.

Popolnoma enaki sklepi kot v preimeru (ref??) pokažejo tudi, da za število $m$, s katerim se blok začne, in $m+1$, ki je drugo število v bloku, spet velja $$\{\sigma_2(1), \sigma_2(1), \ldots, \sigma_2(m) \}=[1,m]_{\Z},$$ oziroma $$\{\sigma_2(1), \sigma_2(1), \ldots, \sigma_2(m+1) \}=[1,m+1]_{\Z},$$ torej je $b_m=1$ in $b_{m+1}=1$.
Spet nam ostane le še vprašanje, kako je z elementi $b_j$, kjer $j$ ni prvo, drugo ali zadnje število v nekem bloku $C_i$. 
Naj bo $m+k$ število, ki ni ne na začetku, ne na drugem mestu in ne na koncu bloka $C_i$ in naj bo $m+k \in [m,m+n]_{\Z}$ oziroma $\sigma_2(m+k) \in \{\sigma_2(m), \sigma_2(m+1), \ldots, \sigma_2(m+n)$. Iščemo $b_{m+k}$. Postopamo podobno kot v primeru (ref??). 
Vemo, da je $$\{ \sigma_2(1), \sigma_2(2), \ldots, \sigma_2(m+k) \}=[1,m-1]_{\Z}\cup \{ \sigma_2(m), \sigma_2(m+1), \ldots, \sigma_2(m+k) \},$$
torej se lahko omejimo na obravnavoi množice $\{ \sigma_2(m), \sigma_2(m+1), \ldots, \sigma_2(m+k) \}.$ Ta vsebuje števila iz bloka $\tilde{C}_i$, torej se v njej izmenjujejo števila iz blokov $C_i^1$ in $C_i^2$. Spet opazimo, da če sta števili $n_1$ in $n_2$ obe vsebovani na primer v bloku $C_i^1$ in je $n_1<n_2$, bo tudi v permutiranem bloku $\tilde{C}_i$ število $n_1$ nastopilo pred številom $n_2$.
Enako kot prej lahko množico vseh števil iz bloka $C_i^1$, ki so vsebovana v množici $\{ \sigma_2(m), \sigma_2(m+1), \ldots, \sigma_2(m+k) \},$ da zapisati kot $[m,m+j]$, kjer je $$m=\min \{C_i^1 \cap \{ \sigma_2(m), \sigma_2(m+1), \ldots, \sigma_2(m+k) \} \}$$
in $$m+j=\max \{C_i^1 \cap \{ \sigma_2(m), \sigma_2(m+1), \ldots, \sigma_2(m+k) \} \}$$
in množica vseh števil iz bloka $C_i^2$, ki so vsebovana v množici $\{ \sigma_2(m), \sigma_2(m+1), \ldots, \sigma_2(m+k) \},$ da zapisati kot $[m+o,m+p]$, kjer je $$m+o=\min \{C_i^2 \cap \{ \sigma_2(m), \sigma_2(m+1), \ldots, \sigma_2(m+k) \} \}$$
in $$m+p=\max \{C_i^2 \cap \{ \sigma_2(m), \sigma_2(m+1), \ldots, \sigma_2(m+k) \} \}.$$
Sledi $$\{ \sigma_2(m), \sigma_2(m+1), \ldots, \sigma_2(m+k) \}=[m,m+j]_{\Z} \cup [m+o,m+p]_{\Z}.$$
Dobimo enak rezultat, kot v primeru (ref??), torej, da je $$\{ \sigma_2(1), \sigma_2(m+1), \ldots, \sigma_2(m+k) \}= [1,m+j]_{\Z} \cup [m+o,m+p]_{\Z},$$
ki predstavlja končni zapis, če je $m+j+1<m+o$. V tem primeru je $m+k$-ti element zaporedja števil blokov $b_{m+k}=2$. Če pa je $m+j+1=m+o$, pa sei zapis poenostavi na $$\{ \sigma_1(1), \sigma_1(m+1), \ldots, \sigma_1(m+k) \}=[1,m+k]_{\Z}$$
in velja $b_{m+k}=1$.
Ker je bila izbira števila $m+k$ poljubna, so torej vsi členi zaporedja števil blokov za permutacijo $\sigma_2$ enaki ali $1$ ali $2$, zaporedje števil blokov je zato omejeno in po izreku (ref??) permutacija $\sigma_2$ ohranja konvergenco.

Potrebujemo le še primer pogojno divergentne vrste, ki jo bo permutacija $\sigma_2$ preuredila tako, da bo konvergirala. Oglejmo si naslednjo vrsto:
\begin{align}\label{eq:10}
\sum_{n=1}^{\infty}a_n=&1 + \left(\frac{1}{4}+\frac{1}{4}-\frac{1}{2} \right) + \left(\frac{1}{6}+\frac{1}{6}+\frac{1}{6}+\frac{1}{6}-\frac{1}{3}-\frac{1}{3} \right)+ \nonumber \\
&+\left(\frac{1}{8}+\frac{1}{8}+\frac{1}{8}+\frac{1}{8}+\frac{1}{8}+\frac{1}{8}-\frac{1}{4}-\frac{1}{4}-\frac{1}{4}\right) \pm \cdots
\end{align}

Zaporedje delnih vsot te vrste je tako:
$$\{S_n\}_{n=1}^{\infty}= 1,\ \frac{5}{4}, \ \frac{6}{4}, 1, \ \frac{7}{6}, \ \frac{8}{6}, \ \frac{9}{6}, \ \frac{10}{6}, \ \frac{8}{6}, \ 1, \ldots$$

Opazimo, da to zaporedje delnih vsot vsebuje podzaporedje, katerega členi so vsi enaki $1$. Torej je $1$ eno izmed stekališč zaporedja $\{S_n\}_{n=1}^{\infty}$. Vidimo pa tudi, da tisti elementi zaporedja $\{S_n\}_{n=1}^{\infty}$, ki imajo indeks $n=b_k$, kjer je $\{b_k\}_{k=1}^{\infty}$ zaporedje, podano z rekurzivno formulo $b_k=b_{k-1}+3 k-1$, $b_1=3$, tvorijo podzaporedje $\{T_n\}_{n=1}^{\infty}$ zaporedja $\{S_n\}_{n=1}^{\infty}$, za katerega velja $$T_n = \frac{2 n-2}{n}.$$ Vidimo, da je limita zaporedja $\{T_n\}_{n=1}^{\infty}$, ko gre $n$ proti neskončno, enaka $2$. Torej je $2$ še eno izmed stekališč zaporedja $\{S_n\}_{n=1}^{\infty}$. Ker ima zaporedje delnih vsot vsaj dve različni stekališči, je divergentno. Sledi, da je tudi vrsta \eqref{eq:10} divergentna.
%tega, da ravno členi z indeksom iz b_n sestavljajo podzaporedje, nikjer ne dokažem. je to narobe?

Če sedaj člene vrste \eqref{eq:10} delujemo s permutacijo $\sigma_2$, dobimo vrsto
\begin{align}\label{eq:11}
\sum_{n=1}^{\infty}a_{\sigma_2(n)}=&1+\frac{1}{4}-\frac{1}{2}+\frac{1}{4}+\frac{1}{6}+\frac{1}{6}-\frac{1}{3}+\frac{1}{6}+\frac{1}{6}-\frac{1}{3}+ \nonumber \\
&+\frac{1}{8}+\frac{1}{8}-\frac{1}{4}+\frac{1}{8}+\frac{1}{8}-\frac{1}{4}+\frac{1}{8}+\frac{1}{8}-\frac{1}{4} \pm \cdots
\end{align}

Da pokažemo, da je dobljena vrsta konvergentna, si spet oglejmo zaporedje delnih vsot:
$$\{\tilde{S}_n\}_{n=1}^{\infty}= 1,\ \frac{5}{4}, \ \frac{3}{4}, 1, \ \frac{7}{6}, \ \frac{8}{6}, \ 1, \ \frac{9}{8}, \ \frac{10}{8}, \ 1,  \ \frac{9}{8}, \ \frac{10}{8}, \ 1 \ldots$$

Vemo, da neko zaporedje $\{a_n\}_{n=1}^{\infty}$ konvergira natanko tedaj, ko konvergira zaporedje $\{a_n\}_{n=m}^{\infty}$, kjer je $m$ neko naravno število. Torej bo zaporedje $\{\tilde{S}_n\}_{n=1}^{\infty}$ konvergiralo natanko tedaj, ko bo konvergiralo zaporedje $\{\tilde{S}_n\}_{n=4}^{\infty}$. Opazimo, da je zaporedje $\{\tilde{S}_n\}_{n=4}^{\infty}$ sestavljeno iz treh podzaporedij. Členi $\tilde{S}_n$, kjer je $n=3j+1$ za $j \geq 0$, tvorijo podzaporedje, katerega členi so vsi enaki $1$. Členi $\tilde{S}_n$, kjer je $n=3j+2$ za $j \geq 0$, tvorijo podzaporedje 
$$\{U_k\}_{k=3}^{\infty}=\frac{7}{6}, \ \frac{7}{6}, \ \frac{9}{8}, \ \frac{9}{8}, \ \cdots, \ \frac{2k+1}{2k}, \ \frac{2k+1}{2k}, \ \cdots,$$
členi $\tilde{S}_n$, kjer je $n=3$ za $j \geq 1$, pa tvorijo podzaporedje 
$$\{V_k\}_{k=3}^{\infty}=\frac{8}{6}, \ \frac{8}{6}, \ \frac{10}{8}, \ \frac{10}{8}, \ \cdots, \ \frac{2k+2}{2k}, \ \frac{2k+2}{2k}, \ \cdots$$
Vsa tri podzaporedja imajo limito $1$, ko gre $k$ proti neskončno. Ker omenjena tri podzaporedja pokrivajo celotno zaporedje $\{\tilde{S}_k\}_{k=4}^{\infty}$, je $1$ edino stekališče oziroma limita tega zaporedja. Sledi, da zaporedje $\{\tilde{S}_k\}_{k=1}^{\infty}$ konvergira, zato je tudi vrsta \eqref{eq:11} konvergentna.
Tako smo dokazali, da je tudi $\sigma_2$ $\lambda$-permutacija.

\end{primer}

\section{Zaključek}



\begin{thebibliography}{99}

\bibitem{vir1}
S.~G.~Krantz in J. D. McNeal, \textit{Creating more convergent series}, Amer.~Math.~Monthly \textbf{111} (2004) 32-38.
\bibitem{vir2}
D.~Velleman, \textit{A note on $\lambda$-permutations}, Amer.~Math.~Monthly \textbf{113} (2006) 173-178.
\bibitem{vir3}
P.~Schaefer, \textit{Sum-preserving rearrangements of infinite series}, Amer.~Math.~Monthly \textbf{88} (1981), 33-40.
\bibitem{vir4}
J.~Globevnik in M.~Brojan, \textit{Analiza I},  DMFA - založništvo, Ljubljana, 2010.

\end{thebibliography}

\end{document}

